\chapter{Introduction} \label{chap:intro}

Throughout human history, collaboration has been an integral part of our pursuit of knowledge. Effective communication and a deep comprehension of each other have propelled us to advance our understanding and technological progress. In the digital age, it is only natural to harness the power of technology to extend collaboration to the digital realm, connecting people around the world. This is particularly pertinent in the aftermath of the COVID-19 pandemic, which has underscored the importance of digital tools in facilitating collaboration.

Virtual Reality (VR) emerges as a compelling platform for collaboration, offering a heightened sense of presence and immersion that inherently enhances social capabilities. A noteworthy subset of VR is DeskVR, immersing users in a virtual environment using a stereoscopic head-mounted display (HMD), all while comfortably seated at an office desk. DeskVR presents unique opportunities for interactions with the virtual environment, focusing on enhancing comfort, reducing physical fatigue, and improving accessibility to VR.

\section{Context and Motivation}

The transition of real-world collaborative tasks to the digital realm demands careful consideration and a thorough understanding of how we communicate. Often, certain tasks become more challenging in the digital space, leading users to prefer and be more efficient in performing those tasks in the real world. During this transition, social information tends to become opaque, limiting users' understanding of each other \cite{ericksonSocialTranslucenceApproach2000}.

VR is an excellent medium for collaboration since users can coexist in a virtual space, visually observing each other's actions and communicating through voice. Still, designing for awareness in VR presents some challenges, particularly with interfaces and information presentation. For example, reading text can be difficult in VR \cite{rauSpeedReadingVirtual2018, kimUserDiscomfortUsing2021}. Nonetheless, advancements in resolution and HMD technology, such as eye gaze tracking, are addressing this issue.

Using VR while standing and relying on mid-air interaction for extended periods can induce fatigue and may be less accessible for individuals with mobility impairments. DeskVR addresses this concern, allowing users to interact with the environment while seated. The use of an office desk offers additional benefits, serving as a surface for touch-based approaches, providing passive haptic feedback, and serving as a virtual representation for presenting social information or operational instructions \cite{zielaskoMenusDeskSystem2019, sousaVRRRRoomVirtualReality2017}. However, constraining users to a seated position limits their movement, requiring special attention while designing for DeskVR.

For instance, one common use for VR is the analysis of 3D models in various industries such as automotive \cite{LAWSON2016323, Zimmermann2008}, urban planning \cite{10.1145/3284389.3284491}, and interior design \cite{10.1145/3605390.3605419}. These tasks often involve rotating and zooming in on models to inspect them from different angles. This can be challenging in traditional VR settings, as users may need to perform complex mid-air movements to manipulate the models. DeskVR users, confined to a desk, have limited mobility and range for such movements. Therefore, it is essential to design object interaction techniques that consider range, physical demand, and ergonomics for seated users \cite{almeidaSIT6IndirectTouchbased2023}.

In collaborative VR settings, effective communication and information sharing are important for enhancing collective understanding and interaction with the virtual environment. For instance, the ability to mark and discuss points of interest can significantly aid in collaborative tasks. This collaboration is important because it enables teams from around the world to work together, share ideas, and make decisions. However, the challenge lies in seamlessly integrating these collaborative features within DeskVR, ensuring that the system is user-friendly, minimizes physical strain, and effectively supports social and workspace awareness.

\section{Objectives}

The goal of this work is to design, implement, and evaluate a collaborative approach for analyzing and communicating about 3D models and their points of interest within DeskVR. This involves a comprehensive examination of current state-of-the-art techniques in collaboration, computer-supported cooperative work (CSCW), and DeskVR, with the objective to identify the difficulties and challenges inherent in these areas and to explore promising avenues for enhancing collaboration and awareness in seated VR environments.

\section{Document Structure}

Chapter \ref{chap:sota}, \textit{Related Work}, explores the current state-of-the-art relevant to this research. It starts with an investigation into computer-supported cooperative work, then examines techniques employed to maintain consistency in shared virtual environments, and subsequently delves into DeskVR interaction techniques. The chapter ends with a discussion of the examined work, relating the concepts between various topics and offering insights into the design of the proposed solution.

Chapter \ref{chap:method}, \textit{Replico}, presents a high-level overview of the design of the proposed solution, Replico, offering detailed specifications and descriptions for each component necessary for the solution's implementation.

Chapter \ref{chap:prototype}, \textit{Implementation of a Prototype}, showcases the concrete implementation following the design standards established in Chapter \ref{chap:method}. It details the technologies and hardware used, as well as delving into more technical aspects.

Chapter \ref{chap:eval}, \textit{Evaluation}, outlines the testing methodology and the subsequent evaluation of the obtained results.

Finally, Chapter \ref{chap:concl}, \textit{Conclusions}, brings this work to a close by revisiting key points, suggesting potential future research directions, and providing concluding remarks.

