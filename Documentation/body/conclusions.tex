\chapter{Conclusions} \label{chap:concl}

This preparatory report for the dissertation explains the significance of fostering collaboration within DeskVR environments. It examines the foundational concepts of collaboration, explores methods to enhance communication in the digital realm, assesses the efficacy of VR in facilitating collaboration, and discusses the advantages and limitations specific to DeskVR.

Through a literature review, three primary themes were encountered. The first revolves around computer-supported cooperative work, investigating how users collaborate using computers. Within this topic, the exploration covered social translucence, workspace awareness, and the spatial model of interactions. The second area focused on concurrency control, which involves techniques used to preserve consistency amidst concurrent updates to objects. Three main approaches for concurrency control were detailed: object ownership, attribute separation, and distributed averages. The third area explored DeskVR interaction techniques, uncovering various methods for interacting with DeskVR environments, such as travel and object manipulation. This exploration provided insights into how to leverage DeskVR's constraints effectively.

The review highlights the potential value of creating replicas to enhance communication and maintain consistency in object arrangements within shared virtual environments in DeskVR. Moving forward, the next steps involve continuous experimentation with technology to refine the proposal, providing a deeper understanding of its requirements. The anticipated challenges encompass designing effective environmental interactions, presenting social information in a comprehensible and unobtrusive manner, and addressing the networking components. These challenges will guide the subsequent phases of research and experimentation.

% audio feedback
% multiple point deletion
% raycast look to to-scale model -> would help with making the to-scale model more used
% way to know own color
% help menu gestures
% ways to make accidental touches less significant
% people liked to see each other in the virtual environment
% teleport long press was confusing
    % in general it was confusing with many steps
% better implementation of the touch frame limits, as some people found it hard to know when it was in the limits
    % fade in from either inside or outside, instead of from the camera's direction
    % much harder to implement
% networking could be improved, for example, the MoveUserToTableClientRpc does not need to exist, instead, whenever the tablenewtork object's seat's change, the user should be moved to the table. Much simpler.
% something nice, the hands should be color coded or even size coded in the virtual touch frame -> for example primary hand are red, secondary hand are blue. in retrospect this makes immense sense, dumb idiot
% BETTER TESTING, especially on the collaborative task