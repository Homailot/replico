\chapter{Conclusions} \label{chap:concl}

    Collaboration is an integral part of human life, and it is essential to understand how to effectively enable it in digital tools to ensure social information is preserved and users remain aware of it. Virtual Reality (VR) holds great promise in this regard, as it facilitates more natural interactions with the digital world and allows users to share the same space. However, prolonged VR use can be physically tiring, limiting user engagement duration. DeskVR addresses this issue by allowing users to remain seated at their desks while fully immersed in a virtual environment, thus reducing the physical strain of VR use. This dissertation aimed to explore and design a collaborative approach that enables seated users to interact with a virtual environment, considering the constraints of a seated position, and to evaluate how this approach impacts user experience and collaboration.

    The literature review explored concepts of social and workspace awareness, concurrency control, and existing work in the field of DeskVR. Although the study on concurrency control did not suggest specific avenues for exploration, it introduced the idea of personal workspaces, leading to an investigation of the world-in-miniature concept. Research in DeskVR highlighted the value of touch-based interactions in enhancing user comfort and reducing the physical strain of VR. The study on social and workspace awareness showed the importance of these concepts in collaboration and identified their key components. With this understanding, the requirements for the proposed approach were defined: to enable users to communicate collaboratively and effectively about objects or areas of interest in 3D models while minimizing physical effort and providing workspace awareness, all while remaining seated.

    This dissertation proposes Replico, a collaborative approach for DeskVR that allows users to communicate about 3D models in a virtual environment using the world-in-miniature metaphor. Replico enables users to explore the 3D model using touch controls to manipulate a personalized miniature replica and to create points of interest using the Balloon Selection metaphor. These points of interest are replicated in both the miniature and the 3D model. Users are anchored to virtual tables representing their real-life counterparts and can join other users' tables to share their perspectives. Additionally, users can teleport around the 3D model to explore it from different angles in true scale. The miniature displays social information about users, such as their positions, and identifies them by appearance. Points of interest are appearance-coded to correspond with the user who created them.

    A prototype was developed using Unity to evaluate the proposed approach. Gesture detection was implemented using a state machine and uses the K-means clustering algorithm to track the user's hands. The system tracked the user's table with a VR controller to align with the real-life table. Various forms of visual feedback, such as finger trails and touch frame limits, were provided to enhance user interaction. Networking functionality was implemented using Unity's Netcode for GameObjects.

    The approach was evaluated through a user study with 20 participants, forming ten pairs. Participants were required to perform five tasks to assess the efficacy and efficiency of several Replico techniques. These tasks were conducted in two scenarios to test the approach's applicability to various 3D models: a large city and a small but detailed Mars Perseverance rover. After each scenario, participants filled out forms based on the NASA-TLX to evaluate the usability of the approach and their experience with it.

    The results showed that participants could efficiently create points of interest in both scenarios, with no significant differences in task completion and active time. Replico effectively notified users of new points of interest, though larger models required more effort from users. The world-in-miniature metaphor proved useful for communicating points of interest, especially in the smaller rover scenario. While users generally found the approach user-friendly and not physically demanding, they struggled with the teleportation gesture, indicating a need for a more intuitive solution.

    These findings suggest that the approach is practical for collaborative interaction in VR, enabling efficient and user-friendly creation and acknowledgment of points of interest. The study highlights the potential and versatility of the world-in-miniature metaphor for enhancing workspace awareness and social information sharing in large, complex environments and smaller, detailed models. Additionally, it demonstrates the value of touch-based interactions and DeskVR in reducing physical strain and improving user comfort. However, the complexity of the environment increases the effort required from users, and the teleportation gesture needs further refinement. These insights point to areas for future improvement and further development.

\section{Future Work}

    The evaluation of Replico revealed several areas for improvement and avenues for future work. One key area for enhancement is the teleportation gesture, which was found to be confusing and difficult to execute. Future work could explore alternative gestures or methods for teleportation that are more intuitive and user-friendly. Specifically, the teleportation orientation gesture could be refined. Research on designing tactile interfaces for older users suggests that touches should not have multiple functions based on duration or speed of press, nor should they include double taps. This principle could be applied to the teleportation gesture, which currently uses a long press to initiate teleportation. Improving finger tracking to be more robust against hardware limitations or making gestures more resilient to faulty tracking and accidental touches could further enhance this feature. Additionally, incorporating a step-by-step interactive guide and a help menu could assist users in learning and utilizing the gestures more effectively.

    Improving feedback on gestures is another area for future work. Color-coding finger trails to provide visual feedback on user actions, similar to the visual feedback in the implementation of SIT6 \cite{almeidaSIT6IndirectTouchbased2023}, could be beneficial. For instance, fingers from one hand could be red, while fingers from the other could be blue. Audio feedback when a gesture is recognized, a finger is detected, or a point of interest is created could also improve user experience, as suggested by studies on auditory feedback in tabletop interfaces \cite{auditoryAwareness}. Haptic feedback, such as vibrotactile feedback when a point of interest is created, could enhance the interaction experience.

    Additionally, the teleportation behavior could be improved by matching the user's head position to the balloon's position, providing a better frame of reference for where the user will be teleported. Currently, it is ambiguous where the user's vertical position will be after teleportation.

    Another point of improvement is addressing the issue of points of interest obscuring the objects they are meant to highlight. Future work could explore making points of interest more transparent or allowing users to adjust their transparency to improve visibility.

    Users also reported difficulty determining when they were within the touch frame limits. Providing more explicit feedback on the touch frame limits, such as fading in from either inside or outside the frame limits, could address this issue. This would require a more complex shader implementation but could be a valuable addition to the system.

    New interactions could be added to the approach to enhance collaboration and communication. For example, users could create zones of interest encompassing an area rather than a single point. Allowing users to simultaneously delete and acknowledge multiple points by increasing the balloon size could also be beneficial. This feature could also be used to create larger or smaller points of interest, prioritizing some points. Additional transformations, such as rotation on other axes, could be introduced. Another interaction could be the ability to lock the replica to prevent accidental movement or reset it to its original position. Users could also hide the replica from sight to focus on the 3D model.

    Other potential interactions include showing the user's gaze direction to help others understand what the user is looking at, which could be achieved by displaying a ray from the user's head to the object they are looking at. With this interaction, users could even create points of interest by looking at an object and performing a gesture. Allowing users at the same table to communicate through audio could further enhance collaboration.

% BETTER TESTING, especially on the collaborative task
    % additional metric: finger movement on transform vs finger movement on balloon selection
    % more careful planning of the tasks and hardware setup
    % test with more users
    % tasks to test the usefullness of teleportation
    % replica scale metric was wrongly calculated