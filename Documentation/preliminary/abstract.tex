%\chapter*{Resumo}


\chapter*{Abstract}

The transition of real-world collaborative and teamwork tasks to the digital world has proven challenging, often resulting in the loss of inherent social aspects of collaboration. Research has addressed this gap by exploring social translucence, an approach to designing systems for social processes by implementing visibility, awareness, and accountability.

Virtual Reality (VR) is an excellent medium for incorporating the principles of social translucence due to its high sense of presence and immersion, which naturally enhance social capabilities. A subset of VR, known as DeskVR, allows users to be fully immersed in a virtual environment while remaining seated at their desks. DeskVR offers a practical solution for extended working hours by reducing the physical fatigue of standing and performing mid-air gestures.

An important application of VR is the analysis of 3D models, used extensively in industries such as automotive, urban planning, and interior design. Collaboration in this context is essential for global teams to work together, share ideas, and make decisions, thereby reducing prototyping costs. This dissertation proposes Replico, a collaborative approach for analyzing and communicating about 3D models using points of interest in DeskVR. Replico employs the world-in-miniature (WIM) metaphor and touch controls to enable users to explore 3D models, and create and discuss points of interest. This approach enhances workspace awareness and social information sharing while minimizing physical strain.

A user study was conducted to evaluate Replico's effectiveness and usability, with 20 participants forming ten pairs. The results indicated that while Replico is generally effective and user-friendly, larger 3D models require more effort, and the teleportation gesture requires refinement. The study highlighted the potential of the WIM metaphor for enhancing workspace awareness and social information sharing in both large, complex environments and smaller, detailed models. Additionally, the study demonstrated the value of touch-based interactions and DeskVR in reducing physical strain and improving user comfort.

\vspace{1em}

\noindent\textbf{Keywords:} gestural input, virtual reality, world-in-miniature, collaboration, social translucence, awareness, DeskVR, points of interest

\vspace{1em}

\noindent\textbf{ACM Classification:}

\begin{itemize}
    \item Human-centered computing $\rightarrow$ Human computer interaction (HCI) \\ $\rightarrow$ Interaction paradigms $\rightarrow$ Virtual reality 
    \item Human-centered computing $\rightarrow$ Human computer interaction (HCI) \\ $\rightarrow$ Interaction paradigms $\rightarrow$ Collaborative interaction
\end{itemize}
