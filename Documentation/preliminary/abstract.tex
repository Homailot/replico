%\chapter*{Resumo}


\chapter*{Abstract}

The transition of real-world collaborative and teamwork tasks to the digital world has proven challenging, often resulting in the loss of the inherent social aspects of real-world collaboration. Research has addressed this gap by exploring social translucence -- an approach to designing systems for social processes by implementing visibility, awareness, and accountability.

Virtual Reality (VR) is an excellent vector for incorporating the principles of social translucence. Due to its high sense of presence and immersion, VR inherently possesses enhanced social capabilities. A subset of VR of particular interest in this domain is DeskVR. DeskVR allows users to be fully immersed in a virtual environment while remaining seated at their desks. It offers a practical solution for extended working hours by reducing physical fatigue associated with standing and mid-air gestures.

Nevertheless, close interaction collaboration, such as the shared manipulation of an object, is complex due to the nature of the system. One fundamental challenge is the lack of awareness regarding other users' intentions. This absence contributes to the unpredictability of concurrent object manipulation, exacerbated by the lack of a physical link between users and the object. Furthermore, devising interaction techniques for DeskVR is difficult due to limited physical mobility and space.

This work proposes a method for implementing social awareness in the context of shared object manipulation in DeskVR. The proposed method involves designing a specialized multi-user interaction technique suited to the constraints of DeskVR. By designing with social translucence in mind, this technique helps users understand each other, fostering harmonious communication and collaboration. The research will also comprise an empirical user study to validate the effectiveness of this approach, evaluating the solution's usability and effectiveness.

\vspace{1em}

\noindent\textbf{Keywords:} virtual environments, virtual reality, shared object manipulation, collaboration, social translucence, awareness, DeskVR

\vspace{1em}

\noindent\textbf{ACM Classification:}

\begin{itemize}
    \item Human-centered computing $\rightarrow$ Human computer interaction (HCI) \\ $\rightarrow$ Interaction paradigms $\rightarrow$ Virtual reality 
    \item Human-centered computing $\rightarrow$ Human computer interaction (HCI) \\ $\rightarrow$ Interaction paradigms $\rightarrow$ Collaborative interaction
\end{itemize}
