\chapter*{Abstract}

The transition of real-world collaborative and teamwork tasks to the digital world has proven challenging, often resulting in the loss of inherent social aspects of collaboration. Research has addressed this gap by exploring social translucence, an approach to designing systems for social processes by implementing visibility, awareness, and accountability.

Virtual Reality (VR) is an excellent medium for incorporating the principles of social translucence and workspace awareness due to its high sense of presence and immersion, which naturally enhance social capabilities. A subset of VR, known as DeskVR, allows users to be fully immersed in a virtual environment while remaining seated at their desks. DeskVR offers a practical solution for extended working hours by reducing the physical fatigue of standing and performing mid-air gestures.

An important application of VR is the analysis of 3D models, used extensively in industries such as automotive, urban planning, and interior design. Collaboration in this context is essential for global teams to work together, share ideas, and make decisions, thereby reducing prototyping costs. This dissertation proposes Replico, a collaborative approach for analyzing and communicating about 3D models using points of interest in DeskVR. Replico employs the world-in-miniature (WIM) metaphor and touch controls to enable users to explore 3D models, and create and discuss points of interest. This approach enhances workspace awareness and social information sharing while minimizing physical strain.

A user study was conducted to evaluate Replico's effectiveness and usability, with 20 participants forming ten pairs. The results indicated that while Replico is generally effective and user-friendly, larger 3D models require more effort, and the teleportation gesture requires refinement. The study highlighted the potential of the WIM metaphor for enhancing workspace awareness and social information sharing in both large, complex environments and smaller, detailed models. Additionally, the study demonstrated the value of touch-based interactions and DeskVR in reducing physical strain and improving user comfort.

\vspace{1em}

\noindent\textbf{Keywords:} gestural input, virtual reality, world-in-miniature, collaboration, social translucence, workspace awareness, DeskVR, points of interest

\vspace{1em}

\noindent\textbf{ACM Classification:}

\begin{itemize}
    \item Human-centered computing $\rightarrow$ Human computer interaction (HCI) \\ $\rightarrow$ Interaction paradigms $\rightarrow$ Virtual reality 
    \item Human-centered computing $\rightarrow$ Human computer interaction (HCI) \\ $\rightarrow$ Interaction paradigms $\rightarrow$ Collaborative interaction
\end{itemize}

\chapter*{Resumo}

A transição das tarefas de colaboração e trabalho de equipa do mundo real para o mundo digital tem-se revelado desafiante, resultando frequentemente na perda dos aspetos sociais inerentes à colaboração. Estudos abordaram esta lacuna através da exploração de "social translucence", uma abordagem à conceção de sistemas para processos sociais através da implementação de visibilidade, consciencialização e responsabilidade.

A Realidade Virtual (RV) é um excelente meio para incorporar os princípios de "social translucence" e "workspace awareness" devido ao seu elevado sentido de presença e imersão, que naturalmente reforçam as capacidades sociais. Um subconjunto da RV, conhecido como DeskVR, permite que os utilizadores estejam totalmente imersos num ambiente virtual enquanto permanecem sentados nas suas secretárias. DeskVR oferece uma solução prática para horários de trabalho alargados, reduzindo o cansaço físico de estar de pé e de fazer gestos no ar.

Uma importante aplicação da RV é a análise de modelos 3D, amplamente utilizada em setores como o automóvel, o planeamento urbano e o design de interiores. A colaboração neste contexto é essencial para que equipas globais possam trabalhar em conjunto, partilhar ideias e tomar decisões, reduzindo assim os custos de prototipagem. Esta dissertação propõe o Replico, uma abordagem colaborativa para analisar e comunicar sobre modelos 3D usando pontos de interesse em DeskVR. O Replico utiliza a metáfora world-in-miniature (WIM) e controlos táteis para permitir aos utilizadores explorar modelos 3D e criar e discutir pontos de interesse. Esta abordagem aumenta a consciência do espaço de trabalho e a partilha de informações sociais, minimizando o esforço físico.

Foi realizado um estudo de utilizadores para avaliar a eficácia e a usabilidade do Replico, com 20 participantes formando dez pares. Os resultados indicaram que, embora o Replico seja geralmente eficaz e fácil de utilizar, os modelos 3D maiores requerem mais esforço e o gesto de teletransporte necessita de aperfeiçoamento. O estudo realçou o potencial da metáfora WIM para aumentar a consciência do espaço de trabalho e a partilha de informação social, tanto em ambientes grandes e complexos como em modelos mais pequenos e detalhados. Além disso, o estudo demonstrou o valor das interacções baseadas no toque e do DeskVR na redução do esforço físico e na melhoria do conforto do utilizador.

\vspace{1em}

\noindent\textbf{Palavras-chave:} entrada gestual, realidade virtual, world-in-miniature, colaboração, social translucence, workspace awareness, DeskVR, pontos de interesse

\vspace{1em}

\noindent\textbf{Classificação ACM:}

\begin{itemize}
    \item Human-centered computing $\rightarrow$ Human computer interaction (HCI) \\ $\rightarrow$ Interaction paradigms $\rightarrow$ Virtual reality 
    \item Human-centered computing $\rightarrow$ Human computer interaction (HCI) \\ $\rightarrow$ Interaction paradigms $\rightarrow$ Collaborative interaction
\end{itemize}

